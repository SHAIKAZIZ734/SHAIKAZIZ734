\usepackage{listings}  % for code listings

% Define custom commands
\newcommand{\solution}{\noindent \textbf{Solution: }}
\newcommand{\myvec}[1]{\ensuremath{\begin{bmatrix} #1 \end{bmatrix}}}
\providecommand{\norm}[1]{\left\lVert#1\right\rVert}

\begin{document}

\begin{center}                                                                     \Large \textbf{Relations and Functions}                                    \end{center}
                                                                               \section{Introduction}
Recall that the notion of relations and functions, domain, co-domain, and range have been introduced earlier.                                                                                                                                \begin{enumerate}                                                                  \item ${(a, b) \in A \times B: a \text{ is brother of } b}$                    \item ${(a, b) \in A \times B: a \text{ is sister of } b}$
    \item ${(a, b) \in A \times B: \text{age of } a \text{ is greater than age of } b}$
    \item ${(a, b) \in A \times B: \text{total marks obtained by } a \text{ in the final examination is less than the total marks obtained by } b}$
    \item ${(a, b) \in A \times B: a \text{ lives in the same locality as } b}$.                                                                              \end{enumerate}
                                                                               \section{Types of Relations}                                                   In this chapter, we will study different types of relations and functions, composition of functions, invertible functions, and binary operations.                                                                                            \begin{enumerate}                                                                  \item Definition 1: A relation \( R \) in a set \( A \) is called an \textbf{empty relation} if no element of \( A \) is related to any element of \( A \), i.e., \( R = \emptyset \subseteq A \times A \).                                  \item Definition 2: A relation \( R \) in a set \( A \) is called a \textbf{universal relation} if each element of \( A \) is related to every element of \( A \), i.e., \( R = A \times A \).                                           \end{enumerate}                                                                                                                                               Both the empty relation and the universal relation are sometimes called trivial relations.

\subsection{Examples}                                                          \begin{enumerate}
    \item \textbf{Example 1:} Let \( A \) be the set of all students of a boys' school. Show that the relation \( R \) in \( A \) given by:
    \[
    R = \{(a, b) : a \text{ is sister of } b\}
    \]
    is the empty relation, and
    \[
    R' = \{(a, b) : \text{the difference between heights of } a \text{ and } b \text{ is less than 3 meters}\}
    \]
    is the universal relation.

    \solution Since the school is a boys' school, no student can be a sister of any other student. Hence, \( R = \emptyset \), showing that \( R \) is the empty relation.

    It is also obvious that the difference between the heights of any two students in the school must be less than 3 meters. This shows that \( R' = A \times A \), which is the universal relation.
\end{enumerate}

\section{Equivalence Relations}
To study equivalence relations, we first consider three important types of relations:

\begin{enumerate}
    \item \textbf{Reflexive:} A relation \( R \) in a set \( A \) is reflexive if \( (a, a) \in R \) for every \( a \in A \).
    \item \textbf{Symmetric:} A relation \( R \) in a set \( A \) is symmetr@@@
star.tex                                                     47,1           60%
