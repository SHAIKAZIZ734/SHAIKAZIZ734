\documentclass[12pt]{article}
\usepackage{graphicx}
%\documentclass[journal,12pt,twocolumn]{IEEEtran}
\usepackage[none]{hyphenat}
\usepackage{graphicx}
\usepackage{listings}
\usepackage[english]{babel}
\usepackage{graphicx}
\usepackage{caption}
\usepackage{hyperref}
\usepackage{booktabs}
\usepackage{array}
\usepackage{amsmath}   % for having text in math mode
\usepackage{listings}
\lstset{
  frame=single,
  breaklines=true
}
%New macro definitions
\newcommand{\mydet}[1]{\ensuremath{\begin{vmatrix}#1\end{vmatrix}}}
\providecommand{\brak}[1]{\ensuremath{\left(#1\right)}}
\providecommand{\norm}[1]{\left\lVert#1\right\rVert}
\newcommand{\solution}{\noindent \textbf{Solution: }}
\newcommand{\myvec}[1]{\ensuremath{\begin{pmatrix}#1\end{pmatrix}}}
\let\vec\mathbf

\begin{document}
\begin{center}
\textbf\large{CHAPTER-1 \\ RELATIONS AND FUNCTIONS}
\end{center}

\section*{1.1 Introduction}

Recall that the notion of relations and functions, domain, co-domain and range have been introduced in Class XI along with different types of specific real valued functions and their graphs. The concept of the term `relation' in mathematics has been drawn from the meaning of relation in the English language, according to which two objects or quantities are related if there is a recognisable connection or link between the two objects or quantities. Let \( A \) be the set of students of Class XII of a school and \( B \) be the set of students of Class XI of the same school. Then some of the examples of relations from \( A \) to \( B \) are:
\begin{itemize}
    \item \( \{(a, b) \in A \times B : a \text{ is brother of } b\} \),
    \item \( \{(a, b) \in A \times B : a \text{ is sister of } b\} \),
    \item \( \{(a, b) \in A \times B : \text{age of } a \text{ is greater than age of } b\} \),
    \item \( \{(a, b) \in A \times B : \text{total marks obtained by } a \text{ in the final examination is less than the total marks obtained by } b\} \),
    \item \( \{(a, b) \in A \times B : a \text{ lives in the same locality as } b\} \).
\end{itemize}

However, abstracting from this, we define mathematically a relation \( R \) from \( A \) to \( B \) as an arbitrary subset of \( A \times B \). If \( (a, b) \in R \), we say that \( a \) is related to \( b \) under the relation \( R \) and we write as \( a \, R \, b \). In general, if \( (a, b) \in R \), we do not bother whether there is a recognisable connection or link between \( a \) and \( b \). As seen in Class XI, functions are a special kind of relations.

In this chapter, we will study different types of relations and functions, composition of functions, invertible functions, and binary operations.

\section*{1.2 Types of Relations}

In this section, we would like to study different types of relations. We know that a relation in a set \( A \) is a subset of \( A \times A \). Thus, the empty set \( \emptyset \) and \( A \times A \) are two extreme relations. For illustration, consider a relation \( R \) in the set \( A = \{1, 2, 3, 4\} \) given by
\[
R = \{(a, b) : a - b = 10\}.
\]
This is the empty set, as no pair \( (a, b) \) satisfies the condition \( a - b = 10 \). Similarly, \( R' = \{(a, b) : |a - b| \geq 0\} \) is the whole set \( A \times A \), as all pairs \( (a, b) \) in \( A \times A \) satisfy \( |a - b| \geq 0 \). These two extreme examples lead us to the following definitions:

\begin{definition}
A relation \( R \) in a set \( A \) is called \textbf{empty relation}, if no element of \( A \) is related to any element of \( A \), i.e., \( R = \emptyset \subset A \times A \).
\end{definition}

\begin{definition}
A relation \( R \) in a set \( A \) is called \textbf{universal relation}, if each element of \( A \) is related to every element of \( A \), i.e., \( R = A \times A \).
\end{definition}

Both the empty relation and the universal relation are sometimes called trivial relations.

\textbf{Example 1:} Let \( A \) be the set of all students of a boys' school. Show that the relation \( R \) in \( A \) given by \( R = \{(a, b) : a \text{ is sister of } b\} \) is the empty relation and \( R' = \{(a, b) : \text{the difference between heights of } a \text{ and } b \text{ is less than 3 meters}\} \) is the universal relation.

\textbf{Solution:} Since the school is a boys' school, no student of the school can be a sister of any student of the school. Hence, \( R = \emptyset \), showing that \( R \) is the empty relation. It is also obvious that the difference between heights of any two students of the school has to be less than 3 meters. This shows that \( R' = A \times A \) is the universal relation.

\textbf{Remark:} In Class XI, we have seen two ways of representing a relation, namely the raster method and the set builder method. However, a relation \( R \) in the set \( \{1, 2, 3, 4\} \) defined by \( R = \{(a, b) : b = a + 1\} \) is also expressed as \( a \, R \, b \) if and only if \( b = a + 1 \) by many authors. We may also use this notation, as and when convenient. If \( (a, b) \in R \), we say that \( a \) is related to \( b \) and we denote it as \( a \, R \, b \).

One of the most important relations, which plays a significant role in Mathematics, is an equivalence relation. To study equivalence relations, we first consider three types of relations, namely reflexive, symmetric, and transitive.

\begin{definition}
A relation \( R \) in a set \( A \) is called:
\begin{itemize}
    \item \textbf{reflexive}, if \( (a, a) \in R \) for every \( a \in A \),
    \item \textbf{symmetric}, if \( (a_1, a_2) \in R \) implies that \( (a_2, a_1) \in R \), for all \( a_1, a_2 \in A \),
    \item \textbf{transitive}, if \( (a_1, a_2) \in R \) and \( (a_2, a_3) \in R \) implies that \( (a_1, a_3) \in R \), for all \( a_1, a_2, a_3 \in A \).
\end{itemize}
\end{definition}

\end{document}
